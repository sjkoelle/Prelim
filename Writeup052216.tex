\documentclass{uwstat572}
%%\setlength{\oddsidemargin}{0.25in} %%\setlength{\textwidth}{6in} %%\setlength{\topmargin}{0.5in} %%\setlength{\textheight}{9in}
\usepackage{amsmath}
\usepackage{url}
\usepackage{epstopdf,gensymb}
\usepackage{graphics}
\usepackage{graphicx}
%\usepackage[numbers]{natbib}
%\usepackage{endfloat}
\usepackage{amsfonts, animate,subfigure}
\usepackage{amsthm,amsmath,amssymb}
%%\usepackage[margin=1in]{geometry}
\usepackage{bbm}
%\usepackage[sc]{mathpazo}
%\linespread{1.05}  
\usepackage{setspace}
\usepackage[export]{adjustbox}
\usepackage{amsmath}
\usepackage{bm}
\usepackage[normalem]{ulem}

\usepackage[usenames,dvipsnames]{xcolor}

\usepackage{mathrsfs}

\usepackage{thmtools}

\renewcommand{\baselinestretch}{1.5}
\bibliographystyle{plainnat}
\renewcommand{\d}[1]{\mathbb{#1}}
\DeclareMathOperator{\sgn}{sgn}
\DeclareMathOperator\erf{erf}
\newcommand{\vmdel}[1]{\sout{#1}}
\newcommand{\vmadd}[1]{\textbf{\color{red}{#1}}}
\newcommand{\vmcomment}[1]{({\color{blue}{VM's comment:}} \textbf{\color{blue}{#1}})}
\newcommand{\skcomment}[1]{({\color{purple}{SK's comment:}} \textbf{\color{purple}{#1}})}
\begin{document} %%\maketitle
\begin{center}  
{\LARGE Square Root Graphical Models: Multivariate Generalizations of
Univariate Exponential Families that Permit Positive Dependencies, by David Inouye, Pradeep Ravikumar, and Inderjit Dhillon}\\\ \\   
 {Samson Koelle \\
    Department of Statistics, University of Washington Seattle, WA, 98195, USA  
     } 
  \end{center}
%%\begin{abstract} 
 %% \end{abstract}
\section{Introduction} 

Understanding relationships between different categorical or quantitative random variables is a central problem in clustering, network reconstruction, and regression.  This paper proposes a multivariate distributions whose natural parameter spaces enable encapsulation of positive conditional dependencies between Poisson random variables.  This is enabled by the speed of the asymptotic decrease of the base measure of the Poisson distribution.  The proposed distribution is also able to encapsulate certain previously described multivariate distributions.

\subsection{Exponential Families}

The univariate exponential family of distributions includes many of the most commonly encountered probability distributions, such as Normal, Poisson, and exponential.   These distributions are of the form
  \[P(x|\theta) = \exp(\eta(\theta)^T T(x) + B(x) - A(\theta)),\]
  where $\eta(\theta)$ are the natural parameters of the exponential family, $T(x)$ is a set of fixed size containing data dependent sufficient statistics, $B(x)$ is the base measure of the distribution, and 
  \[A(\theta) = \log \int_{-\infty}^{\infty} \eta(\theta)^T T(x) + B(x) dx\]
  is the normalizing constant of the distribution.  These distributions both encapsulate diverse behavior and have a number of useful properties.  For example, they have sufficient statistics of fixed size which fully summarize data, often appear in physical and information theoretic systems, and arise as unique maximum entropy distributions given constraints on moments.  For these reasons and many others, an exponential family distribution is often chosen to model univariate data.
  
 \subsection{Graphical Models}
 
Graphical models provide a general framework for constructing multivariate probability distributions.  A graph $G$ is composed of $p$ sites and $e$ edges which link pairs of sites.  The Hammersley-Clifford Theorem states that we can construct a multivariate probability measure that satisfies the Markov independence property encoded by $G$ if and only if its density factorizes as
 %%\[\mathbb{P}(x) = \frac{1}{Z(\theta)} \prod_{c \in \mathcal{C}}\phi_c(x) \qquad \phi_c:\mathbb{S}|_c \to \mathbb{R} \]
 \[\mathbb{P}(x) = \frac{1}{Z(\theta)} \prod_{c \in \mathcal{C}}\phi_c(x) \]
 where $\mathcal{C}$ are the cliques of the graph, i.e. simplicial combinations of non-zero conditional probability dependencies and $\phi_c$ are statistics which summarize the distribution of the data on sites in the clique.  Since we often understand what univariate exponential family distribution with which to model a particular parameter, the calculation of compatibility functions from univariate node-conditional exponential family distributions is therefore the crucial issue when constructing multivariate distributions over a graph. \vmcomment{include proper citations in this paragraph.}

  
 
\subsection{Normalizability}

\subsection{Previous Work}
This approach in fact has its root in node-wise regression methods for learning these models  (Meinshausen
and B?uhlmann, 2006; Ravikumar et al., 2010; Jalali et al., 2011), in which for each site $s$ in a graph $G{p,e}$ with $p$ vertices, the interaction terms \vmcomment{``interaction terms" are not defined yet} $\Phi_{-s} \in \mathbb{R}^{p-1}$ are learned through the regression $X_s = X_{-s} \Phi_{-s}$\vmadd{,} where $X_s \in \mathbb{R}^n$.  \vmcomment{Try to avoid math in the intro} Since we often assume sparsity in our edge set, it can be desirable to limit the number of edges through use of either a information theoretic method like BIC, or a penalization.  Previous work has focused on learning these models in the multivariate Gaussian, multivariate Ising, and discrete settings, as well as mixed settings in which certain variables are Gaussian and others are Ising \citep{Friedman2008-oq, LauWer89, Bento2011-vw}.   Other attempts to create a multivariate Poisson distribution have been unsuccessful for alternate reasons such computational complexity, restrictions to positive-interactions, or unnatural transforms applied to sufficient statistics (Inouye 2013, 11-16).

\section{Methods}

In this section, I describe the SRGM distribution, its normalizability, and its implementation through convex model fitting and stochastic integration of the normalizing constant.  The general set up used throughout the rest of the paper is as follows: given data $\mathbb{X}$ composed of a set of $n$ i.i.d. observations of $p$ parameters.  Vectors $\bm{x}$ in bold refer to single observations of all parameters, and are assumed to be i.i.d.

\subsection{Background}

Since we often understand what univariate exponential family distribution with which to model a particular parameter, we assume that node-conditional distributions are from a univariate exponential family whose natural parameters are a linear combination of k-th order interaction terms as follows: 
\[P(x_s \vert x_{-s}, \Theta_{\mathcal{C}_s}) = \exp{ ( \sum_{c \in \mathcal{C}_s} (\theta_c \prod_{i \in c} T(x_i) ) + B(x_s) - A(\Theta_{\mathcal{C}_s} )) }    \]
 where $\mathcal{C}_s$ is the set of cliques containing site $s$ and $\Theta_{\mathcal{C}_s}$ is the set of clique parameters $\theta_c$ for site $s$.  We can see that this is a univariate exponential family whose natural parameter depends on conditional dependencies by factoring $T(x_s)$ out of the product term.  By the Hammersley-Clifford Theorem, these conditionals specify the unique joint distribution
 \[P(\bm{x} \vert \Theta) = \exp{ (\sum_{s} ( \sum_{c \in \mathcal{C}_s} (\theta_c \prod_{i \in c} T(x_i) ) + B(x_s)) - A(\Theta ))} . \]
 Restricting to clique sizes of at most two, we arrive at the general multivariate exponential family model proposed by Yang (2012) \skcomment{Yang claimed that this was the unique joint distribution for node-conditional exponential families, but this is obviously not the case, since the power of the interaction terms has been changed in the SRGM}
 \[P(\bm{x} \vert \Theta^1,\Theta^2) = \exp(\Theta^{1T} T(x) + T(x)^T\Theta^{2}  T(x) + \sum_{s=1}^p B(x_s) - A(\Theta))\]
 where $\Theta^1$ and $\Theta^2$ represent clique-specific parameters of orders $1$ and $2$ respectively in the case that $\Theta^2$ is all zeroes. Note that this form encapsulates the multivariate normal distribution through the parameterizations $\Theta^1 = \Sigma^{-1} \mu$ and $\Theta^2 = -\frac{1}{2}\Sigma^{-1}$ \skcomment{in this case, theta2 will not have all zero diagonals... this is a minor contradiction with the paper}, as well as independent Poisson and exponential distributions when $\Theta^{2} = 0$.

\subsection{Square Root Graphical Models}

A weakness of the previous model is that the pairwise interaction terms $T(\bm{x})^T \Theta^{2} T(\bm{x})$ are $O(x^2)$ while the self-interaction terms $T(\bm{x})^T \Theta^{1}$ are $O(x)$.  Positive terms in $\Phi$ will \skcomment{again, I am not sure why negative definiteness would not be an acceptable criteria for the Poisson distribution}.  Accordingly, the authors propose the following \textit{Square Root Graphical Model} (SRGM)
\[P(\bm{x} \vert \Theta, \Phi) =  \exp(\Theta^T \sqrt{T(\bm{x})} + \sqrt{T(\bm{x})}^T\Phi \sqrt{T(\bm{x})} + \sum_{s=1}^p B(x_s) - A(\Theta,\Phi)).\]

Note that the diagonals of $\Phi$ are $\Theta^1$ from the previous parameterization.  
To recover the multivariate Gaussian, we set entry-wise $T(x) = x^2$ and replace $\sqrt{T(\bm{x})}$ with $\sgn{(\bm{x})}\sqrt{T(\bm{x})}$ in the above expression, and use the same reparameterization as above.  However, the SRGM differs from the previous model in several ways.  
\[P(x_s \vert x_-s, \Phi, \Theta) = \exp{(\eta_{1s}^{node} \sqrt{x_s} + \eta_{2s}^{node} x_s + B(x_s) - A(\eta_{1s}^{node} , \eta_{2s}^{node} ))} \]
where $\eta_{1s}^{node} = \Theta_s + 2 \Phi_{-s}\sqrt{x_{-si}}$ and $\eta_{2s}^{node} = \Phi_{ss}$.  For base exponential families which are one-parameter, this reduces to the base family in the case $\eta_{1s} = 0$.  This contrasts with previous constructions.



\subsection{Normalizability of Square Root Graphical Models}

The SRGM has a natural parameter space  
\[\mathcal{N} = \{ (\Theta,\Phi)  \vert  \int \exp(\Theta^T \sqrt{T(\bm{x})} + \sqrt{T(\bm{x})}^T\Phi \sqrt{T(\bm{x})} + \sum_{s=1}^p B(x_s)) d\mu (x) < \infty \} \]
which depends on the sufficient statistics and base measures of the conditional probability distributions of the sites which make up the base graph, as well as their individual natural parameter spaces. The inclusion of positive dependencies in the natural parameter space of the SRGM follows from the normalizability of the radial distributions of $\bm{\|\bm{x}\|_1}$ given $v = \frac{x}{\|x\|_1}$, which are the two-parameter exponential families
 \begin{align*}
P(\bm{x} = z\bm{v} \vert \bm{v}, \Phi, \Theta) &=  \exp{(\Theta^T \sqrt{T(z \bm{v})} +  \sqrt{T(z \bm{v})} \Phi \sqrt{T(z \bm{v})} + \sum_{s=1}^p B(zv_s))} \\
&= \exp{( (\Theta^T \sqrt{\bm{v}}) \sqrt{z} +  (\sqrt{\bm{v}^T} \Phi \sqrt{\bm{v}})z  + \sum_{s=1}^p B(zv_s))} \\
&= \exp{(\eta_{1}^{rad} \sqrt{z} + \eta_{2}^{rad} z + \sum_{s=1}^p B(x_s) - A(\eta_{1}^{rad} , \eta_{2}^{rad}))} 
\end{align*}
where $\eta_{1}^{rad} = \Theta^T \sqrt{\bm{v}}$,  $\eta_{2}^{rad} = \sqrt{\bm{v}^T} \Phi \sqrt{\bm{v}}$, and $T(x) = x$, as is the case in the Poisson and Exponential distributions \skcomment{they claim that this can be assumed w.l.o.g. but I feel like this is not correct}.
Rewriting the joint normalizing constant in terms of the radial distribution, we get
\begin{align*}
A(\Phi, \Theta) &= \log \int_\mathcal{D} \exp{(\Theta^T \sqrt{T(\bm{x})} +  \sqrt{T(\bm{x})} \Phi \sqrt{T(\bm{x})} + \sum_{s=1}^p B(x_s))}d\mu (\bm{x}) \\
			&= \log \int_\mathcal{V} \int_{\mathcal{Z}(\mathcal{V})} \exp{(\Theta^T \sqrt{T(z \bm{v})} +  \sqrt{T(z \bm{v})} \Phi \sqrt{T(z \bm{v})} + \sum_{s=1}^p B(zv_s))} d\mu(z)d\bm{v} 
\end{align*}
where $\mathcal{D}$ is the domain of the probability measure, $\mathcal{V} =\{\bm{v} \vert \|\bm{v}\|_1 = 1, \bm{v} \in \mathcal{D} \}$, and $\mathcal{Z}(\mathcal{V}) = \{z \in \mathbb{R}_+ \vert z\bm{v} \in \mathcal{D} \}$.
Since $\mathcal{V}$ is bounded, we need to show only that the interior integral is finite, i.e.
\[A^{rad} (\eta_1, \eta_2) = \int \exp{(\eta_{1}^{rad} \sqrt{z} + \eta_{2}^{rad} z + \sum_{s=1}^p B(x_s))} d\mu(z) < \infty, \]
which is the case if $\eta_2 < 0$ or $\eta_2 = 0$, and $\eta_1 \leq 0$.  This condition, which is automatically satisfied if $\Phi$ is negative definite, may be relaxed for the Poisson distribution because of the rapid asymptotic decrease of its base measure $B_{pois}(x) = \log \frac{1}{x!}$.  \skcomment{I am unclear why positive or negative definiteness of Phi was not sufficient in the previous class of pairwise exponential family MRFs detailed by Yang in "Graphical Models via Univariate
Exponential Family Distributions"}

\subsection{Selection of Model Parameters}
Given data $\mathbb{X}$ containing $n$ observations of $p$ parameters, the authors seek to estimate the model parameters $\Phi \in \mathbb{R}^{p\times p}$ and $\Theta \in \mathbb{R}^p$  that maximize the joint likelihood 
 \[L(\Theta,\Phi | \mathbb{X}) = \prod_{i=1}^{n} \exp(\Theta^T\sqrt{T(x)} + \sqrt{T(x)}^T\Phi \sqrt{T(x)} - A(\Theta,\Phi)).\]
As a computationally tractable substitute for maximizing this joint likelihood, they minimize the sum of the penalized node conditional negative log likelihood joint objective functions, which is
 \[O(\eta_1 , \eta_2 | \mathbb{X}) = \sum_{s = 1}^{p} (-\frac{1}{n} \sum_{i = n}^n (\eta_{1si} \sqrt{x_{si}} + \eta_{2si} x_{si} - A_{node} ( \eta_{1si}, \eta_{2si} )) + \lambda\| \Phi_{-s} \|_1).\]
Here, $\eta_{1si} = \Theta_s + 2 \Phi_{-s}\sqrt{x_{-si}}$ and $\eta_{2si} = \Phi_{ss}$.  Parameter optimization via proximal gradient descent requires solving for the gradients
\begin{align*}
\frac{dO}{d\phi_{ss}} &= \frac{-1}{n} \sum_{i=1}^{n} x_{si} - \frac{dA_{node}}{d\eta_2}, \\
\frac{dO}{d\phi_{-s}} &= \frac{-2}{n} \sqrt{\mathbb{X}_{-s}}^T \sqrt{\mathbb{X}_s} - 2 \sqrt{\mathbb{X}_{-s}} \frac{dA_{node}}{d\eta_1}\, \quad \text{and}  \\
\frac{dO}{d\theta_{s}} &= \frac{-1}{n} \sum_{i=1}^{n} \sqrt{x_{si}} - \frac{dA_{node}}{d\eta_1}.
\end{align*}
as well as approximating the gradients $\frac{dA_{node}}{d\eta1}$ = $\frac{(A_{node}((\eta_1 + \epsilon), \eta_2) - A_{node}((\eta_1 , \eta_2)))}{\epsilon} $ using $\epsilon = 0.0001$, and $\frac{dA_{node}}{d\eta_2}$ similarly.  While $A_{node}$ was approximated in the Arxiv paper through Gaussian quadrature, the authors have now computed an analytic solution
\[A_{node} = \sqrt{\pi}  \eta_1 \exp{ ( \frac{\eta_1^2}{-4\eta_2} ) } / (2(-\eta_2)^{1.5}) (1 - \erf{(\frac{-\eta_1}{2 \sqrt{ -\eta_2}}}))  + \frac{1}{-\eta_2}\].
The optimization algorithm then proceeds via line-search and soft-thresholding.


\subsection{Estimation of Normalization Constant}
Given model parameters $\Theta$ and $\Phi$, we use Annealed Importance Sampling to estimate the normalizing constant $A(\Theta, \Phi)$.  AIS requires a function $f_0$ proportional to a target distribution $P_0$, an easy to sample from normalized distribution $P_n$, a series of functions  $f_{n-1} \dotsc f_1$ proportional to a series of distributions $P_{n-1} \dotsc P_1$ which interpolate $P_n$ and $P_0$, and a mechanism $T_j$ such as Gibbs sampling or Metropolis Hastings for generating samples from $P_j$ using $f_j$.  In particular, the AIS algorithm proceeds as follows:

	Generate $x_{n-1}$ from $P_n$
	
	Generate $x_{n-2}$ from $x_{n-1}$ using $T_{n-1}$
	
	$\dotsc$
	
	Generate $x_0 $ from $x_1 $ using $T_1$

For each particle path $x = \{x_n, \dotsc x_0\}$, we compute the weight $w(x) = \prod_{i=1}^n \frac{f_{i-1}(x_{i-1})}{f_{i}(x_{i-1})}$.  The average of the sample weights converges to the ratio of the normalizing constants of $P_0$ and $P_n$.  This procedure is essentially a repeated application of importance sampling using MCMC to sample from intermediate distributions.
	In our case, 2000 independent particles are simulated from a $P_0$ which is an independent exponential family with parameters determined by the current iterate.  We create $n=200$ intermediate distributions $P_i = P(\Phi_{diag} + \frac{n-i}{n} (\Phi - \Phi_{diag}), \frac{n-i}{n}{\Theta} $, and make each $T_{j}$ a sequence of $20$ Metropolis-Hastings steps.  
	
	
 \section{Results} 
 \section{Discussion}


 When the sufficient statistic is a linear function $T(X) = X$, then the order two multivariate exponential family is in fact a generalized linear model, and this linear effect node-conditional model can be extended to graphical models in which each site in the graph may have a probability density corresponding to different exponential families, such as Bernoulli and Normal.
 The arumIn order to ensure that the resulting multivariate distribution is normalizable, the natural parameter space can be restricted to negative or positive definite matrices, but in cases where the base measure decreases with the sufficient statistic, this condition may be relaxed because the decrease in base measure cancels out increases in sufficient statistics.  The Poisson distribution has the quickly decreasing base measure $\frac{1}{x!}$, which enables .  In particular, 
 
 On the other hand, if univariate marginal distributions are not exponential families, and therefore (by the theorem of Pitman-Koopman-Darmois) do not have bounded sets of node-specific sufficient statistics, then the number of potential interactions between which will explode since the dimensionalities of the sufficient statistics are each themselves growing.
Dependencies between parameters are often interesting in their own right or as part of the regression problem of optimizing $\theta$ in $Y \approx \mathbb{X}\theta$ where $\mathbb{X} = \{X_1, \dotsc X_n\}$.  \vmcomment{You are operating with mathematical notation without defining it; very hard to understand what you are trying to say} 
 Sufficiency is a poorly understood notion in the multivariate case.
 Note that inferring dependencies locally is also a regression problem $X_a \approx \mathbb{X}_{-a}\theta_a$, where $\mathbb{X}_{-a}$ are adjacent sites of parameter $a$ and $\theta_a$ are their edge weights \citep{Meinshausen2006-kn}. 
 %%Inouye 2013
 %%11 S. L. Lauritzen. Graphical models, volume 17. Oxford University Press, USA, 1996.
%%[12] I. Yahav and G. Shmueli. An elegant method for generating multivariate poisson random variable. Arxiv
%%preprint arXiv:0710.5670, 2007.
[%%13] A. S. Krishnamoorthy. Multivariate binomial and poisson distributions. Sankhya: The Indian Journal of �
%%Statistics (1933-1960), 11(2):117?124, 1951.
%%[14] P. Holgate. Estimation for the bivariate poisson distribution. Biometrika, 51(1-2):241?287, 1964.
%%[15] D. Karlis. An em algorithm for multivariate poisson distribution and related models. Journal of Applied
%%Statistics, 30(1):63?77, 2003.
[%%16] N. A. C. Cressie. Statistics for spatial data. Wiley series in probability and mathematical statistics, 1991
%Square Root Graphical Models, proposed by Inouye et al \vmcomment{use citet here instead of hardcoding the first author name}, dominate previous multivariate exponential family proposals, and enable construction of interactions that were previously precluded.

  \bibliography{stat572}
   \end{document}